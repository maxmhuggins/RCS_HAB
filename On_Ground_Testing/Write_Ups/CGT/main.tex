\documentclass[letterpaper,12pt]{article}
\usepackage[table,xcdraw]{xcolor}
\usepackage[version=4]{mhchem}
\usepackage{tabularx} % extra features for tabular environment
\usepackage{amsmath}  % improve math presentation
\usepackage{graphicx} % takes care of graphic including machinery
\usepackage{wrapfig}
\usepackage{tcolorbox}
\tcbuselibrary{skins,breakable}
\usepackage[margin=1in,letterpaper]{geometry} % decreases margins
\usepackage{cite} % takes care of citations
\usepackage[final]{hyperref} % adds hyper links inside the generated pdf file
\hypersetup{
	colorlinks=true,       % false: boxed links; true: colored links
	linkcolor=blue,        % color of internal links
	citecolor=blue,        % color of links to bibliography
	filecolor=magenta,     % color of file links
	urlcolor=blue         
}
\usepackage{listings}
\usepackage[utf8]{inputenc}
\usepackage{float}
\usepackage{color}
\usepackage[final]{pdfpages}
\definecolor{codegreen}{rgb}{0,0.6,0}
\definecolor{codegray}{rgb}{0.5,0.5,0.5}
\definecolor{codepurple}{rgb}{0.58,0,0.82}
\definecolor{backcolour}{rgb}{0.95,0.95,0.92}
 
\lstdefinestyle{mystyle}{
    backgroundcolor=\color{backcolour},   
    commentstyle=\color{codegreen},
    keywordstyle=\color{magenta},
    numberstyle=\tiny\color{codegray},
    stringstyle=\color{codepurple},
    basicstyle=\footnotesize,
    breakatwhitespace=false,         
    breaklines=true,                 
    captionpos=b,                    
    keepspaces=true,                 
    numbers=left,                    
    numbersep=5pt,                  
    showspaces=false,                
    showstringspaces=false,
    showtabs=false,                  
    tabsize=2
}
 
\lstset{style=mystyle}



\begin{document}

\title{Characterizing a Cold Gas Thruster System}
\author{Max Huggins – UCA Department of Physics and Astronomy}
\date{\today}
\maketitle


\begin{abstract}
A cold gas thruster (CGT) system was designed with pre-existing nozzle theory in mind. This paper deals with characterizing the system and provides an analysis of the force production for it. It was found that the CGT performed similarly to the predicted theory, but the data collected was not sufficient to characterize the system as was previously expected. This required the use of a more general analysis scheme.
\end{abstract}


\section{Introduction}
A cold gas thruster (CGT) is a system that uses expanding gas or gases to generate a force. This force is typically used in reaction control systems (RCSs) to stabilize space craft or simply change their attitude. This paper is primarily concerned with reaction control systems to be developed for high altitude balloons (HABs.) These HABs experience intense and sporadic winds. Winds which make data collection for certain sensors difficult. There are several ways in which a RCS can achieve stabilization, but the method of choice here is the CGT.\\
There are several components important to the CGT RCS. Here, there will only be a brief discussion on these components such that the analysis is not lacking information. The first consideration to make is the type of gas to be used. The primary question here is, \textit{what makes one gas better than another?} and there are excellent parameters that characterize a gas. 

\begin{figure}[!h]
\centering
\includegraphics[scale=.5]{example}
\caption{example}
\label{figure 1}
\end{figure}\newline




\section{The Data and Analysis}




\section{Results of Analysis}




\section{Conclusion}




\section{Nomenclature}



\section{References}



\end{document}
