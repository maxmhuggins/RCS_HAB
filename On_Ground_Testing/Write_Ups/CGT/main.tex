\documentclass[letterpaper,12pt]{article}
\usepackage{amsmath}  % improve math presentation
\usepackage{graphicx} % takes care of graphic including machinery
\usepackage[margin=1in,letterpaper]{geometry} % decreases margins
\usepackage{cite} % takes care of citations
\usepackage{listings}
\usepackage[utf8]{inputenc}
\usepackage[stretch=10]{microtype}



\begin{document}

\title{Characterizing a Cold Gas Thruster System}
\author{Max Huggins – UCA Department of Physics and Astronomy}
\date{\today}
\maketitle


\begin{abstract}
A cold gas thruster (CGT) system was designed with pre-existing nozzle theory in mind. This paper deals with characterizing the system and provides an analysis of the force production for it. It was found that the CGT performed similarly to the predicted theory, but the data collected was not sufficient to characterize the system as was previously expected. This required the use of a more general analysis scheme.
\end{abstract}


\section{Introduction}
A cold gas thruster (CGT) is a system that uses expanding gas to generate a force. This force is typically used in reaction control systems (RCSs) to stabilize space craft or simply change their attitude. This paper is primarily concerned with reaction control systems to be developed for high altitude balloons (HABs.) These HABs experience intense and sporadic winds. Winds which make data collection for certain sensors difficult. There are several ways in which a RCS can achieve stabilization, but the method of choice here is the CGT.\\
There are several components important to the CGT RCS. Here, there will only be a brief discussion on two of these components such that the analysis is not lacking information. \\
The first consideration to make is the type of gas to be used. The primary question here is, \textit{what makes one gas better than another?} One parameter that tries to answer this question is the \textit{specific impulse} ($I_{sp}$). This is a value specific to a gas. Experimentally, it is measured by integrating a force (F) versus time (t) plot generated by a CGT using that gas. That will give the total impulse, this is divided by the change in weight of the gas through that time period ($\tau$). In other words:
\begin{equation}
	I_{sp} = \frac{\displaystyle\int\limits_{t=0}^{\tau}F(t)dt}{mg}
\end{equation}
where m is mass and g is acceleration due to gravity. This is an excellent start to creating a standard for comparing gases, but there is much more that should be considered. Factors such as safety, availability, cost, energy storage density, and so on all contribute to the choice of gas. Additionally, each one of these factors has a different weight per say depending on the scenario in which they are being applied. After consideration, the choice of gas for this system is $CO_2$.\\
The other component is the nozzle. These are used to set the mass flow rate for the system and to accelerate the propellant to supersonic speeds. They consist of a converging section that leads into a throat of minimum radius and a diverging section. Nozzles that have been made to achieve maximum thrust obey the following characteristics. In the converging section, the gas is accelerated; once the throat is reached the gas is traveling at the speed of sound. Then, as it proceeds into the diverging section, it continues to accelerate until it reaches the exit plane of the nozzle. Ideally, the nozzle is designed such that the pressure of the gas exiting the nozzle is equal to the ambient pressure.
%\begin{figure}[!h]
%\centering
%\includegraphics[scale=.5]{example}
%\caption{example}
%\label{figure 1}
%\end{figure}\newline




\section{The Data and Analysis}




\section{Results of Analysis}




\section{Conclusion}




\section{Nomenclature}



\section{References}



\end{document}
