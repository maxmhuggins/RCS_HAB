The over arching goal of this research project is to develop a reaction control system for high altitude balloons. The work completed includes an analysis on viability of the system, an experimental design, data collection, and analysis on it. The experimental design had errors associated with it and because of this the analysis did not provide system characterization. Next steps include correcting the experimental errors, finishing the analysis, integrating the CGT into a HABP, and flying the system. Additional thoughts on future work are the possibility of attempting another axis of stabilization. In this project, only two thrusters were ever really considered. This is because the payload does not see many perturbations in the upward or downward direcitons due to the nature of the balloon system. The biggest consideration here is not just fuel limitations but also the weight limitations. The heaviest part of the system is by far the solenoid; to add another axis of stabilization means adding two more solenoids. \\
In correcting the experimental errors, the options are not left open. There are clear solutions that can be proceeded with quickly. After this though, the project is much more open-ended. Actually integrating the cold gas thruster system into the high altitude balloon payload is not clearly defined here. There are several interesting methods to go about implementing this as an effective reaction control system and the carryover is quite similar to the systems used to stabilize rockets, satellites, etc. Because of this there is paramount information available for proceeding. The many applications of the RCS show its robust nature and the possibilities with it.