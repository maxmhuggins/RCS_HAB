The objective of this section is to derive the relationships used for the analysis of the data collection. First, the general thermodynamic relationships will be established, then they will be applied to a flowing gas. This, ultimately leads to the use of a converging-diverging nozzle. At this point, the relevant relationships for the data collection will be established and the experimental design can be discussed in detail.
\section{General Thermodynamic Relationships}
The theory used here is mostly taken from pre-existing nozzle theory. The source for this project is reference \cite{langton}. Much of the derivation for the equations to be used are based on the following assumptions and rely on Classical Thermodynamics:
\begin{enumerate}
\item The propellant is homogeneous throughout the nozzle.
\item The propellant behaves like a perfect gas.
\item There is no friction between the propellant and the nozzle walls
\item The system is adiabatic
\item The propellant flow is one-dimensional
\item The propellant flow is stagnate in the nozzle chamber
\item The propellant velocity is uniform arccos any cross-section normal to the nozzle axis
\end{enumerate}
Reference \cite{langton} refers to three other assumptions, but they either do not apply or are redundant for the CGTS. Additionally, all of the variables are clearly defined in the nomenclature section but will not be defined along the discussion in this section.\\
Starting with the first law of thermodynamics (FLT),
\begin{equation}\label{eq:FLT}
dQ=dU+PdV
\end{equation}%
\nomenclature{$Q$}{Heat put into a system}%
\nomenclature{$U$}{Internal energy}%
\nomenclature{$V$}{Volume}%
\nomenclature{$P$}{Pressure}%
\nomenclature{$X_x$}{Represents some variable, X, at some point x}%
\nomenclature{$X_i$}{Represents some variable, X, at some initial state}%
\nomenclature{$X_f$}{Represents some variable, X, at some final state}
\nomenclature{$X_s$}{Represents some variable, X, at stagnation}%
\nomenclature{$X_c$}{Represents some variable, X, in the nozzle chamber}%
\nomenclature{$X_t$}{Represents some variable, X, in the nozzle throat}%
\nomenclature{$X_e$}{Represents some variable, X, in exit plane of the nozzle}%
and according to the second law,
\begin{equation}\label{eq:SLT}
dQ=TdS
\end{equation}%
\nomenclature{$T$}{Temperature}%
\nomenclature{$S$}{Entropy}
defining the specific heats at constant volume and pressure respectively,
\begin{equation}\label{eq:SHV}
c_V=\left(\frac{\partial Q}{\partial T}\right)_V
\end{equation}
\begin{equation}\label{eq:SHP}
c_P=\left(\frac{\partial Q}{\partial T}\right)_P
\end{equation}
and using Joule's Equation to define a perfect gas
\begin{equation}\label{eq:Joule}
\left(\frac{\partial U}{\partial V}\right)_T=0
\end{equation}
substituting \ref{eq:FLT} into $\partial Q$ in \ref{eq:SHV}
\begin{equation}
c_V=\left(\frac{\partial U}{\partial T}\right)_V
\end{equation}
Also, substituting \ref{eq:SLT} into $\partial Q$ in \ref{eq:SHV}
\begin{equation}
c_V=T\left(\frac{\partial S}{\partial T}\right)_V
\end{equation}
so
\begin{equation}\label{eq:cV}
c_V=\left(\frac{\partial U}{\partial T}\right)_V=T\left(\frac{\partial S}{\partial T}\right)_V
\end{equation}
similarly with $c_P$
\begin{equation}\label{cP}
c_P=\left(\frac{\partial U}{\partial T}\right)_P=P\left(\frac{\partial V}{\partial T}\right)_P
\end{equation}
this means \ref{eq:SLT} can be written as
\begin{equation}\label{eq:NewSLT}
dQ=c_VdT+PdV
\end{equation}
Considering a perfect gas, the equation of state is
\begin{equation}\label{eq:IGL}
PV=nRT
\end{equation}%
\nomenclature{$R$}{Universal gas constant}%
\nomenclature{$n$}{Number of moles}
substitutting $n=m/W$ provides%
\nomenclature{$W$}{Molecular weight of the gas}
\begin{equation}
PV=\frac{mRT}{W}
\end{equation}
so the equation for 1 unit of mass is
\begin{equation}\label{eq:IGL1}
PV=\frac{RT}{W}
\end{equation}
differentiating $PV$ gives
\begin{align}
d(PV)&=VdP+PdV\\
&=\frac{R}{W}dT
\end{align}
so
\begin{equation}\label{eq:NewSLTinIGL}
PdV=\frac{R}{W}dT-VdP
\end{equation}
Defining a new constant, $\gamma$
\begin{equation}\label{eq:Gamma}
\gamma=\frac{c_P}{c_V}
\end{equation}%
\nomenclature{$\gamma$}{Ratio of specific heats}
from this, \ref{eq:NewSLTinIGL}, \ref{eq:NewSLT}, and \ref{eq:SHP}
\begin{equation}\label{eq:GammaincV}
c_V=\frac{R}{W(\gamma-1)}
\end{equation}
similarly,
\begin{equation}\label{eq:GammaincP}
c_P=\frac{\gamma R}{W(\gamma-1)}
\end{equation}
It can also be determined for an adiabatic system, $dQ=0$, that
\begin{equation}\label{eq:adiabat}
P_iV_i^{\gamma}=P_fV_f^{\gamma}
\end{equation}
From the previous relations and the definition of enthalpy,
\begin{equation}\label{eq:Enthalpy}
H=U+pV
\end{equation}%
\nomenclature{$H$}{Enthalpy}
the enthalpy per unit mass can now be expressed as
\begin{equation}\label{eq:Enthalpypermass}
H=\frac{\gamma RT}{W(\gamma-1)}
\end{equation}
\section{The Nozzle}\label{sec:TheNozzle}
Until this point, all of these relations are quite general. Figure \ref{fig:Nozzle} displays a typical converging diverging nozzle. It consists of three important regions. The furthest left is the chamber, it is here that we assume the variables to be stagnate. Stagnate meaning the velocity of the gas is zero. The next section going in the $+x$ direction is the nozzle throat. This is the smallest cross-sectional area of the system. Next is not a particularly important point, but represents the variables at any point, x, in the nozzle. Lastly is the exit plane of the nozzle. 
\begin{figure}[h!]
\centering
\includegraphics[scale=1]{Figures/Nozzle}
\caption{Simplified nozzle, with axis and some variables defined.}
\label{fig:Nozzle}
\end{figure}
To apply the previous definitions to a nozzle, the kinetic energy of 1 unit of mass will be considered along with the enthalpy. If the flow of the gas is considered, the enthalpy will increase by the amount equal to the kinetic energy of the gas per 1 unit of mass, given the gas exchanges no heat with the environment
\begin{align}
KE&=\frac{v^2}{2}\\
H_x&=\frac{\gamma RT_x}{W(\gamma-1)}\\
H_f&=H_x + KE
\end{align}%
\nomenclature{$KE$}{Kinetic energy}
A region where the gas is stagnant ($v=0$) can be considered, here the enthalpy is a constant. Substitutting \ref{eq:IGL1} and the specific volume ($V=1/\rho$) as well,
\begin{align}\label{eq:CasH}
H_s=C&=\frac{\gamma P_s}{W\rho_s(\gamma-1)}\\
&=\frac{\gamma}{(\gamma-1)}\frac{P_x}{W\rho_x}+\frac{v_x^2}{2}
\end{align}%
\nomenclature{$C$}{A constant relevant to the enthalpy at stagnation}
Now, considering an adiabatic system
\begin{equation}\label{eq:Adiabat}
\frac{V_s}{V_x}=\left(\frac{P_x}{P_s}\right)^{\frac{1}{\gamma}}
\end{equation}
From \ref{eq:CasH} and \ref{eq:Adiabat} it can be found that
\begin{equation}\label{eq:1.20}
\frac{P_x}{P_s}=\left(1-\frac{v^2}{2C}\right)^{\frac{\gamma}{\gamma-1}}
\end{equation}
\subsection{Mach Number}
An important parameter is the Mach number, which is defined as the ratio of the gas velocity at some point to the velocity of sound in the gas. Here, this will be manipulated to determine the parameter in terms of temperature variables.
\begin{align}\label{eq:DefineMach}
M_x&=\frac{v_x}{v_{sound}}\\
&=\frac{v_x}{\sqrt{\gamma R T_x}}
\end{align}%
\nomenclature{$M$}{Mach number}
This can be substitutted into the previously defined relationships to find
\begin{equation}\label{eq:1.25}
\frac{T_s}{T_x}=1+M_x^2\frac{\gamma-1}{2}
\end{equation}
From \ref{eq:SHP} and the fact that the net heat change is equal to the change in kinetic energy per unit mass, the
\begin{align}
dQ&=c_PdT\\
&=c_P(T_x-T_{x+dx})\\
&=\frac{(v_{x+dx}^2-v_x^2)}{2J}
\end{align}
Where $J$ is defined as
\begin{equation}
J=\frac{W_k}{Q}
\end{equation}%
\nomenclature{$W_k$}{Work}
\nomenclature{$J$}{The mechanical equivalent of heat}
Substitutting \ref{eq:GammaincP}, using the stagnation conditions for the initial values, and solving for $v_x$,
\begin{equation}\label{eq:GasVelocity}
v_x=\sqrt{\frac{2\gamma RT_s}{(\gamma-1)W}\left(1-\left(\frac{P_x}{P_s}\right)^{\frac{\gamma-1}{\gamma}}\right)}
\end{equation}
Now, the velocity of the gas at any point x is in terms of variables in regions where flow is stagnated. The temperature can be found similarly,
\begin{equation}\label{eq:GasTemp}
T_x=T_s-\frac{v_x^2}{2Jc_P}
\end{equation}
substituting \ref{eq:GasVelocity} and \ref{eq:GasTemp} into the Mach number
\begin{equation}\label{eq:MachInTermsofT}
M^2=\frac{2}{(\gamma-1)}\left(\frac{T_s}{T_x}-1\right)
\end{equation}
Lastly,
\begin{equation}\label{eq:MRatio}
\frac{M_{x+dx}}{M_x}=\frac{v_{x+dx}}{v_x}\sqrt{\frac{T_x}{T_{x+dx}}}
\end{equation}
\subsection{Area Ratio}
The expansion ratio is the single most important parameter when designing an efficient nozzle. This is the ratio of the exit plane area to the throat plane area. To introduce the variable, the mass flow rate will be defined.
\begin{equation}\label{eq:MassFlow}
w=\frac{dm}{dt}
\end{equation}
In terms of the geometry of the nozzle, the velocity of the gas, and the density
\begin{equation}
w=A_xv_x\rho_x
\end{equation}%
\nomenclature{$\rho$}{Density}
The mass flow rate must be constant throughout any given time in the system. This is the equation of continuity for the flow. The ratio of any two points for a unit mass can now be analysed,
\begin{equation}\label{eq:MassFlowwDensity}
\frac{A_{x+x}}{A_x}=\frac{V_{x+dx}v_x}{V_xv_{x+dx}}
\end{equation}
Using \ref{eq:Adiabat} for the general case, \ref{eq:MachInTermsofT}, and \ref{eq:MRatio}
\begin{equation}
\frac{A_{x+dx}}{A_x}=\frac{M_x}{M_{x+dx}}\left(\frac{T_x}{T_{x+dx}}\right)^\frac{1}{2}\left(\frac{1+\frac{(\gamma-1)}{2}M_{x+dx}^2}{1+\frac{(\gamma-1)}{2}M_x^2}\right)^{\frac{1}{\gamma-1}}
\end{equation}
The expansion ratio then can be expressed as
\begin{align}
\epsilon=\frac{A_e}{A_t}=\frac{M_t}{M_e}\left(\frac{T_t}{T_e}\right)^\frac{1}{2}\left(\frac{1+\frac{(\gamma-1)}{2}M_e^2}{1+\frac{(\gamma-1)}{2}M_t^2}\right)^{\frac{1}{\gamma-1}}
\end{align}%
\nomenclature{$\epsilon$}{Expansion ratio}
To determine the best value for the expansion ratio, the mass flow rate will be maximized and using previously found relationships the value for the expansion ratio will be found. It is here, that the converging-diverging nozzle is found to be the necessary geometry for the nozzle. Starting with substituting \ref{eq:GasVelocity} into \ref{eq:MassFlowwDensity}.
\begin{equation}\label{eq:WforAll}
w=\frac{A_x}{V_x}\sqrt{\frac{2\gamma RT_s}{(\gamma-1)W}\left(1-\left(\frac{P_x}{P_s}\right)^{\frac{\gamma-1}{\gamma}}\right)}
\end{equation}
and
\begin{equation}\label{eq:TtoP}
\left(\frac{T_x}{T_s}\right)^{\frac{1}{\gamma-1}}=\left(\frac{P_x}{P_s}\right)^{\frac{1}{\gamma}}
\end{equation}
Also,
\begin{equation}\label{eq:VtoT}
\frac{1}{V_x}=\frac{1}{V_s}\left(\frac{T_x}{T_s}\right)^{\frac{1}{\gamma-1}}
\end{equation}
substituting \ref{eq:VtoT} and \ref{eq:TtoP} into \ref{eq:WforAll} and rearranging
\begin{equation}\label{eq:diffW}
w=\frac{A_x}{V_s}\sqrt{\frac{2\gamma KT_s}{(\gamma-1)}\left(\left(\frac{P_x}{P_s}\right)^{\frac{2}{\gamma}}-\left(\frac{P_x}{P_s}\right)^{\frac{\gamma+1}{\gamma}}\right)}
\end{equation}
This can now be maximized by differentiating with respect to $P_x$ and set equal to zero. This gives
\begin{equation}
\frac{dw}{dP_x}=\frac{A_x}{2V_s}\left[\frac{2\gamma KT_s}{(\gamma-1)}\left(\left(\frac{P_x}{P_s}\right)^{\frac{2}{\gamma}}-\left(\frac{P_x}{P_s}\right)^{\frac{\gamma+1}{\gamma}}\right)\right]^{-\frac{1}{2}}*\frac{d}{dP_x}\left(\left(\frac{P_x}{P_s}\right)^{\frac{2}{\gamma}}-\left(\frac{P_x}{P_s}\right)^{\frac{\gamma+1}{\gamma}}\right)=0
\end{equation}
Obviously, 
\begin{equation}
\left(\left(\frac{P_x}{P_s}\right)^{\frac{2}{\gamma}}-\left(\frac{P_x}{P_s}\right)^{\frac{\gamma+1}{\gamma}}\right)\neq0
\end{equation}
so
\begin{equation}
\frac{d}{dP_x}\left(\left(\frac{P_x}{P_s}\right)^{\frac{2}{\gamma}}-\left(\frac{P_x}{P_s}\right)^{\frac{\gamma+1}{\gamma}}\right)=0
\end{equation}
differentiating and solving for $P_x/P_s$
\begin{equation}
\frac{P_x}{P_s}=\left(\frac{2}{\gamma+1}\right)^{\frac{\gamma}{\gamma-1}}
\end{equation}
Now, substituting \ref{eq:TtoP} it can be seen that
\begin{equation}
\frac{T_x}{T_s}=\frac{2}{\gamma+1}
\end{equation}
which when compared to \ref{eq:1.25} it is seen that they are equal when 
if $M=1$, and if $x=t$, then the optimum value for $P_t$ can be determined. When $M=1$ at the throat is not necessarily the only condition which maximizes $w$ but it is also a condition which satisfies maximum force production of the nozzle. This can be seen quite easily. 
\subsection{Force}\label{subsec:Force}
If the assumption is made that the conglomerate of particles exiting the nozzle act as a rigid body with some collective velocity and momentum, then it can be said the force generated by that body is
\begin{equation}
F_{gas}=\frac{d(mv_e)}{dt}
\end{equation}
Also, the exhaust velocity is considered to be constant with time, so
\begin{align}
F_{gas}&=\frac{dm}{dt}v_e\\
&=wv_e
\end{align}
Other forces acting on the nozzle are due to the pressure differences of the exit plane and ambient pressures, where $P=F/A$,
\begin{equation}\label{eq:EffectiveVelocity}
F=wv_e + (P_e-P_a)A_e
\end{equation}
\begin{tcolorbox}[breakable,title=Unit Discrepancy, height fixed for=first and middle]
Reference \cite{langton} discusses the discrepancy between the units typically displayed for the specific impulse values and the definition of the specific impulse. This slight commentary on the subject can be skipped without loss of understanding for the relevant theory. Quoted from N.H. Langton's Space Research and Technology Volume 2: Rocket Propulsion, ``Very frequently, rocket engineers quote specific impulse in units of `seconds', but it should be noted that this is rather a loose practice... A good deal of this confusion could have been avoided if the practice of the old rocket pioneers had been maintained, of using the `effective exhaust velocity' ($v_e^{'}$) as their basic parameter! If a consistent system of units of force and mass were employed, then the S.I. [specific impulse] would be numerically equal to $v_e^{'}$. In passing it may be observed that most rocket vehicle performance problems can be handled more easily in terms of momentum and impulse, rather than energy'' The effective velocity referred to is derived from Equation \ref{eq:EffectiveVelocity}.
\begin{align*}
F&=wv_e + (P_e-P_a)A_e\\
&=wv_e^{'}
\end{align*}
\end{tcolorbox}
Here it can be seen that maximizing $w$, maximizes the force produced by the nozzle. Substituting Equation \ref{eq:WforAll} and \ref{eq:GasVelocity}
\begin{equation}\label{eq:TheoreticalForce}
F= A_tP_s\sqrt{\frac{2\gamma^2}{\gamma-1}\left(\frac{2}{\gamma+1}\right)^{\frac{\gamma+1}{\gamma-1}}\left(1-\left(\frac{P_e}{P_s}\right)^{\frac{\gamma-1}{\gamma}}\right)}+\left(P_e-P_a\right)A_e
\end{equation}
From this, the $I_{sp}$ can also be determined using \ref{eq:SpecificImpulse}. In a similar method, the following expression is found
\begin{equation}\label{eq:TheoreticalSpecificImpulse}
I_{sp}=\left[\frac{2\gamma R T_s}{(\gamma-1)W}\left(1-\left(\frac{P_e}{P_s}\right)^{\frac{\gamma-1}{\gamma}}\right)\right]^{\frac{1}{2}}+\frac{(P_e-P_a)}{P_s}\frac{A_e}{A_t}\frac{1}{C_D}
\end{equation}
where $C_D$ is the discharge coefficient and is defined by Equation \ref{eq:DischargeCoefficient}.
\begin{equation}\label{eq:DischargeCoefficient}
C_D=\sqrt{\frac{\gamma \rho_c}{P_s}}\left(\frac{2}{\gamma+1}\right)^{\frac{\gamma+1}{2(\gamma-1)}}
\end{equation}%
\nomenclature{$C_D$}{Discharge coefficient}
\section{Relationships}\label{sec:Relationships}
Looking at Equation \ref{eq:TheoreticalSpecificImpulse} and \ref{eq:TheoreticalForce} it can be seen that the variables are not easily measured. It would be simpler to measure the temperature values instead. A change of variables can be performed plugging \ref{eq:adiabat} and \ref{eq:MachInTermsofT} into them. The following expressions are obtained:
\begin{equation}\label{eq:RelevantNozzleForce}
F = A_t P_c \left(\sqrt{\frac{2\gamma^2}{\gamma-1}\left(\frac{2}{\gamma+1}\right)^{\frac{\gamma+1}{\gamma-1}}\left(1-\frac{T_e}{T_c}\right)}+\left(\left(\frac{T_e}{T_c}\right)^{\frac{\gamma}{\gamma-1}}-\frac{P_a}{P_c}\right)\epsilon\right)
\end{equation}
\begin{equation}\label{eq:RelevantIsp}
I_{sp}=\left[\frac{2\gamma R T_c}{W(\gamma-1)}\left(1-\frac{T_e}{T_c}\right)\right]^{\frac{1}{2}}+\left[\left(\frac{T_e}{T_c}\right)^{\frac{\gamma}{\gamma-1}}-\frac{P_a}{P_c}\right]\frac{A_e}{A_t}\sqrt{\frac{P_c}{\gamma\rho_c}}\left(\frac{\gamma+1}{2}\right)^{\frac{\gamma+1}{2(\gamma-1)}}
\end{equation}
These equations are the basis for this experiment. In determining values for these variables the theoretical and experimental values can be compared. Let $I_{sp}^*$ denote the theoretical specific impulse and $I_{sp,m}$ denote the measured specific impulse. As these are functions of $\epsilon$ ($I_{sp}^*(\epsilon^*)$) it could be found a proportionality between $I_{sp}^*$ and $I_{sp,m}$ such that
\begin{equation}
I_{sp}^*(\epsilon^*)=I_{sp,m}(\epsilon_m*e_{ff})
\end{equation}%
\nomenclature{$I_{sp,m}$}{Measured specific impulse}%
\nomenclature{$\epsilon_m$}{Area ratio yielding maximum force for the measured value}%
\nomenclature{$I_{sp}^*$}{Theoretical specific impulse}%
\nomenclature{$\epsilon^*$}{Area ratio yielding maximum force for the theoretical value}%
\nomenclature{$I_{sp}^*$}{Theoretical specific impulse}%
where $e_{ff}$ denotes some efficiency factor. The same procedure can be done for the force equations. Determining this would allow for the specify the optimum expansion ratio for this system at various ambient pressures.