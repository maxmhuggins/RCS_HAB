Unfortunately, due to COVID-19 the necessary corrections could not be made to the project. This means the system could not be characterized with the intended scheme. It can be seen, though, that the values found are close to the known values from reference \cite{anis}. The theoretical value provided there is $67\ s$ and the measured value is $61\ s$. Included in Table \ref{table:Isps} is a \% difference between the literature's theoretical value (\%Diff. T) and the average value determine for each geometry as well as a \% difference for the literature's experimental value and average value found (\%Diff M). \% differences as low as 1\% and as high as 36\% can be seen there. There are two reasons the results cannot be clearly compared. The first is the $I_{sp}'s$ dependence on temperature and pressure. Meaning that these values are meaningless without the theoretical values at the same pressure and temperature. Reference \cite{anis} is using values for the maximum possible $I_{sp}$ values. The second reason is that there was no error analysis performed here.\\
However, the importance of this research is the development of the system and  scripting. The focus on both readability and generalizability is the value here. Additionally, the work done here has allowed the realization of certain errors. The solutions to these errors are presented below.\\
The most obvious problems with this setup are shown in the analysis. The exit plane temperature is not well represented and neither is the mass flow rate. Additionally, the stagnation conditions for the actual system are not well-defined.
\section{Hardware}
Here, the problems regarding the hardware of the project are discussed.
\subsection{ADC}
The analog to digital converter is used to convert the analog voltage signals from the variable sensors to digital signals that the Raspberry Pi can read in. The resolution of the device is defined by the number of \textit{bits} it has. The MCP3008 is a 10-bit ADC used for the temperature sensors. Since it is a 10-bit ADC, the number of levels ($N$) available for storing voltage signals is 1024. In other words,
\begin{equation}
N=2^{bits}
\end{equation}
The HX711 is a 24-bit ADC, meaning $N=16777216$. The HX711 has the capability of being 16384x more accurate than the MCP3008. The poor resolution is clearly seen in the raw temperature plots, such as the one shown in Figure \ref{fig:BadTemp}. The reason the HX711 was not used for the temperature data, however was the open source library interfacing with it was meant to be used specifically for the load cell. The chip was also on a breakout board and did not have other channel options. Most of the 24-bit ADCs require the use of a standard communication protocol, I2C. Finding open source libraries for these higher resolution devices is difficult and the user would likely have to build one themselves. However, this is not completely out of reach as there are many sources providing information on the subject and implementations in Python do exist. This was not a complete necessity though and was put aside for future design iterations.
\subsection{Pressure Regulator}
The stagnation conditions discussed in Chapter \ref{chap:Theory}, were not well defined prior to the first run of trials. Because of this, data that is meant to be stagnant is not. Also, since the $CO_2$ was spent so quickly it caused some of it to fuse into a solid. This is definitely not ideal for several reasons. The first is that solid $CO_2$ can block plumbing and cause issues for the RCS. Next, is the previously mentioned issue of stagnation. If the $CO_2$ changes temperature so rapidly, the chamber pressure will also be decreasing rapidly. This creates an ill-defined stagnation condition and an optimum nozzle geometry is impossible to correctly determine. With the use of a pressure regulator the stagnation conditions could be clearly determined. Also, this could allow the use of a feedback system that would add heat to the system such as to condition the temperature to stay constant even through a continuous discharge. It may also be possible to create the same effect by decreasing the nozzle throat radius. Either way, this should most likely be done because the mass flow rate for the system is too high.
\section{Software}
Changes to the software include a better method for stopping each experiment. The current method as discussed in Chapter \ref{chap:Analysis} is to determine when the force value dropped below a certain threshold for a specified amount of time. The force sensor sometimes lost its zero position and created problems for stopping data collection. A more robust method would be using time derivatives of the force to classify when the data collection should end. This would create more consistent data files and plots. The scripts should also be updated for the addition of the new hardware mentioned above. Most importantly, is to begin an error analysis section. This would consist of new hardware scripts to determine the reliability of sensors and scripts performing error analysis on the actual functions used. This would not only allow the comparison between other literature values and the determined ones, but it would also allow the system a higher level of reliability. For example, determining how much fuel should be taken for mission objectives could be determined with higher confidence.