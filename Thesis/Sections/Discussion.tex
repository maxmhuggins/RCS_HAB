Unfortunately, no serious results were obtained from the data collected in this project. However, the importance of this research is the development of the system and  scripting. The focus on both readability and generalizability is the value here. Additionally, the work done here has allowed the realization of certain errors. The solutions to these errors are presented below.\\
The most obvious problems with this setup are shown in the analysis. The exit plane temperature is not well represented and neither is the mass flow rate. Additionally, the stagnation conditions for the actual system are not well-defined.
\section{Hardware Problems}
Here, the problems regarding the hardware of the project are discussed.
\subsection{ADC}
The analog to digital converter is used to convert the analog voltage signals from the variable sensors to digital signals that the Raspberry Pi can read in. The resolution of the device is defined by the number of \textit{bits} it has. The MCP3008 is a 10-bit ADC used for the temperature sensors and the , which means the number of levels ($N$) available for storing voltage signals is 1024. In other words,
\begin{equation}
N=2^{bits}
\end{equation}
The HX711 is a 24-bit ADC, meaning $N=16777216$. The HX711 has the capability of being 16384x more accurate than the MCP3008. The poor resolution is clearly seen in the raw temperature plots, such as the one shown in figure \ref{fig:BadTemp}. The reason the HX711 was not used for the temperature data, however was the open source library interfacing with it was meant to be used specifically for the load cell. The chip was also on a breakout board and did not have other channel options. Most of the 24-bit ADCs require the use of a standard communication protocol, I2C. Finding open source libraries for these higher resolution devices is difficult and the user would likely have to build one themselves. However, this is not completely out of reach as there are many sources providing information on the subject and implementations in Python do exist. This was not a complete necessity though and was put aside for future design iterations.
\subsection{Pressure Regulator}
The stagnation conditions discussed in chapter \ref{chap:Theory}, were not well defined prior to the first run of trials. Because of this, data that is meant to be stagnant is not. Also, since the $CO_2$ was spent so quickly it caused some of it to fuse into a solid. This is definitely not ideal for several reasons. The first is that solid $CO_2$ can block plumbing and cause issues for the RCS. Next, is the previously mentioned issue of stagnation. If the $CO_2$ changes temperature so rapidly, the chamber pressure will also be decreasing rapidly. This creates an ill-defined stagnation condition and an optimum nozzle geometry is impossible to correctly determine. With the use of a pressure regulator the stagnation conditions could be clearly determined. Also, this could allow the use of a feedback system that would add heat to the system such as to condition the temperature to stay constant even through a continuous discharge. It may also be possible to create the same effect by decreasing the nozzle throat radius. Either way, this should most likely be done because the mass flow rate for the system is too high.\\